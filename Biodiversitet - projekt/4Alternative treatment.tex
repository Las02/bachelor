\section{Alternative treatment options for \emph{P. aeruginosa} infections} \label{Alternative treatment}

Due to increasing antibiotic resistance in \emph{P. aeruginosa} strains, novel methods of dealing with the pathogen are needed. Two of the most significant contributors to the antibiotic resistance is the formation of biofilm, which can block external compounds, as well as the efflux system. Therefore methods targeting biofilm formation or the zinc homeostasis could be speculated as alternatives to antibiotics. 

\subsection{Targeting zinc homeostasis}
Biofilm formation could be targeted indirectly by inhibiting the systems maintaining zinc homeostasis. This would lower \emph{P. aeruginosas} pathogenicity, since swimming and swarming motility depend on the flagella, which in turns depends on free zinc ions. An example of this was demonstrated by Mastropasqua \emph{et. al} \cite{Mastropasqua2018EfficientLung}. They proved the flagella synthesis is negatively regulated by low zinc concentrations. This in turns impaired the swarming mobility for \emph{P. aeruginosa}, which would decrease the pathogenicity \cite{Mastropasqua2018EfficientLung}.\\

\noindent A challenge posed by inhibiting the zinc homeostasis, could be the large amount of regulatory systems for the import of zinc. Here one inhibited system might just be replaced by the up-regulation of another one.

Another way of targeting the zinc homeostasis, is by the zinc uptake pathway. Since permeability of zinc in the biofilm is low, the bacteria uses metallophores as chelating agents to transport zinc into the cells. This opens a possibility to bio-engineer metallophores to carry antimicrobials instead of zinc. These bio-engineered compounds can be applied to biofilm, to carry antimicrobial agents straight into the cell \cite{Mastropasqua2018EfficientLung}.

\subsection{Photodynamic therapy}
Another promising approach to dealing with bacterial biofilm, is the use of Photodynamic Therapy (PDT). The basis of PDT, is using non-toxic photosensitizers (PS) which are light sensitive compounds. When the PS are illuminated with visible light, it reacts with oxygen and creates reactive oxygen species (ROS) \cite{Darabpour2016ChitosanStudy}. ROS can act on a myriad of targets spanning from biofilm matrices, cell membranes, cellular organelles as well as macromolecules \cite{Darabpour2016ChitosanStudy}. Different ROS can also be very specific in which targets they interact with, thus making them ideal for treatment options. Furthermore, the ROS can be immobilized onto nanoparticles which can improve drug delivery into biofilm \cite{Darabpour2016ChitosanStudy}. An advantage to PDT is the unlikeliness for bacteria to evolve resistance. This is due to the PS not needing to enter the cell to cause damage. Therefore cells can't block the uptake or increase detoxification \cite{Tavares2010AntimicrobialTreatment}.

\subsection{Photothermal therapy}           
In Photothermal Therapy (PTT), light is again utilized to combat bacterial infections, but in a different way. In PTT near infrared light is emitted at nanomaterials, which creates extreme localized heat zones that causes damage to the biofilm. Different nanomaterials can be used for this, ranging from carbon based to gold or silver particles \cite{Korupalli2017Acidity-triggeredInfection}. This type of treatment was demonstrated by Al-Bakri \emph{et. al}, who used gold nanorods (AuNR) and phospholipid-polyethylene (PEG). This resulted in a $\sim$6 log cycle reduction in viable count compared to the reference \cite{Al-Bakri2019Photothermal-inducedBiofilm}. This method also has the possibility to be combined with phage therapy.

\subsection{Phage therapy}
In phage therapy, bacteriophages are used to infect and lyse cells. The main advantage of using phages is their specificity for particular bacterial species. Although bacteria can become resistant to phages, this often entails  a trade-off between the cells. This trade-off cause the cells to lose their antibiotic resistance \cite{Chan2018PhageAeruginosa}. This is partly because their lyseing mechanism is distinct from those of antibiotics \cite{Chan2018PhageAeruginosa}. To increase efficiency of treatment, phage therapy can be combined with PTT. This however is still experimental, and isn't seen in clinical trials. 


