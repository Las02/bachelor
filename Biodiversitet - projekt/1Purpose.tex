\section{Background and purpose}
\textit{P. aeruginosa} is a gram-negative opportunistic pathogen, which is commonly found in a variety of different places such as soil and water \cite{Chatterjee2017EnvironmentalAeruginosa}.
Its large genome makes it able to host a wide range of genes for the adaptation to live under different conditions. It can be found both in its planktonic state and in its biofilm producing state. Biofilm formation can happen on both biotic surfaces and abiotic surfaces \cite{Laborda2021PseudomonasOrigin}. 
Especially on medical devices \textit{P. aeruginosa} contamination can cause serious problems as it can cause nosocomial infections. This problem is further propelled by the fact that \textit{P. aeruginosa} can survive for long stretches of time on very dry places, and sometimes even survive alcohol disinfection as reviewed in Pachori \textit{et. al} \cite{Pachori2019EmergenceReview}\cite{Laborda2021PseudomonasOrigin}.
\textit{P. aeruginosa} mainly infects patients who are immunocompromised such as cystic fibrosis patients. It was estimated that in 2017 32,600 people in the US where infected with \textit{P. aeruginosa}, and around 2,700 died from the infection \cite{MULTIDRUG-RESISTANTAERUGINOSA}. Infection with \textit{P. aeruginosa} are often extremely hard to treat, as the bacteria is resistant to the most commonly and widespread treatments options such as antibiotics \cite{Laborda2021PseudomonasOrigin}. 
Two of the major factors for \textit{P. aeruginosa}'s antibiotic resistance, are biofilm and various efflux pumps \cite{Laborda2021PseudomonasOrigin}. The efflux system, which is a series of membrane bound proteins, can export molecules from inside the cell to outside. This can enable the cell to export anti-microbial compounds from inside cell, thereby increasing resistance \cite{Perron2004CzcR-CzcSAeruginosa}. 
 
Biofilm on the other hand also promotes antibiotic resistance, but by blocking antibiotic access to the cells. Furthermore, the biofilm can support communities of bacteria, which in turn could harbor other pathogenic bacteria. These communities can also act as hot spots for gene transfer, thereby forwarding antibiotic resistance genes to other species, further complicating the challenges antibiotic resistance pose \cite{Madigan2022BrockMicroorganisms}.

Based on these factors \textit{P. aeruginosa} has been listed by the World Health Organization (WHO) as one of the most life-threatening pathogenic bacteria\cite{PRIORITIZATIONTUBERCULOSIS}. It is therefore highly relevant to find possible non-antibiotic treatments for \textit{P. aeruginosa} infections.\\ 
The purpose of this paper is to describe novel ways of treating \textit{P. aeruginosa} infections without the use of antibiotics. To do this the project will focus on \textit{P. aeruginosa}'s systems for maintaining zinc homeostasis. They are relevant since 1) they make it possible for \textit{P. aeruginosa} to survive both low- and high zinc concentration which can be present during infection 2) many of \textit{P. aeruginosa}'s virulence factors are zinc dependent as reviewed in \cite{Gonzalez2019PseudomonasPathogen}. By inhibiting \textit{P. aeruginosa}'s virulence factors, one could possibility decrease \textit{P. aeruginosa's} pathogenicity while minimizing the selective pressure on resistance. Furthermore the project will focus on \textit{P. aeruginosa}'s ability to form biofilm, as it is one of the key virulence factors in \textit{P. aeruginosa} infections.
