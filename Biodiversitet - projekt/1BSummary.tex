\section{Summary}
In this project two virulence factors of the gram-negative bacteria \textit{Pseudomonas aeruginosa} will be presented. These are relevant as \textit{P. aeruginosa} is an opportunistic pathogen and its infections can be lethal. The two virulence factors discussed in this project includes the biofilm formation and a specific efflux pump. These two virulence factors are both to a point zinc dependent. The efflux pump is directly zinc dependent, whereas the biofilm formation depends on zinc through various factors such as the flagella construction. 

Maintaining zinc homeostasis is important for the survival of \textit{P. aeruginosa}. This makes the bacterium able to survive high and low zinc concentrations present during infection. Further it triggers the inhibition of an specific operon (\textit{oprD}), which is responsible for the influx of carbapenems. 

Biofilm formation in itself is hard to treat as it is almost impossible to penetrate with antibiotics. This is due to the biofilm matrix consisting of EPS's. Further, quorum sensing helps initiate biofilm formation. 

As antibiotics is not an efficient option on its own, alternative treatments will be discussed. These alternative treatments includes targeting of the zinc homeostasis, Photodynamic, Photothermal, and Phage Therapy. Although none of these treatments are ready for clinical trials. 

In conclusion \textit{P. aeruginosa} itself and the alternative treatments must be further researched, as there are no final solutions yet. This is partly due to the differences in research and clinical trials. 
