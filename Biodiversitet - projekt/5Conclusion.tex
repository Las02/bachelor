\section{Conclusion}
\emph{P. aeruginosa} is listed as a priority pathogen for Research and Development of new antibiotics by the World Health Organization. This is due to the antibiotic resistant phenotype shown by \emph{P. aeruginosa}, as well as the possibility of HGT of the antibiotic resistance genes to other species of bacteria. The resistance to antibiotic can be attributed to efflux pumps in the cells and biofilm formation. As mentioned previously, the biofilm is indirectly zinc dependent while some efflux pumps are directly depended.  

Zinc homeostasis is regulated in different ways, depending on whether there is a depletion or excess of zinc. If there is a depletion of zinc, the cells can acquire the metal from uptake through membrane transport proteins, metallophores, or by creating paralogous proteins. These paralogous proteins can replace zinc requiring proteins in order to free the metal. If an excess of zinc is present the efflux system is activated, where membrane proteins pump excess zinc out of the cells. 

Another virulence factor of \textit{P. aeruginosa} is the zinc dependent biofilm, which consists of polysaccharides, eDNA and proteins. The sticky matrix from the biofilm helps the bacteria attach to different surfaces, and also protects the bacteria against extrinsic stress like antibiotics. The biofilm can include multiple different species of bacteria, from where HGT can occur. This further complicate the challenge of antibiotic resistance. In biofilm communities, the cells often regulate behaviour by quorum sensing. Some genes are only expressed when larger numbers of colonies are present, to improve fitness. 

Due to the antibiotic resistant phenotype of \emph{P. aeruginosa}, other methods are being researched as alternative treatment options. As described, the zinc uptake is vital for the pathogenesis of \emph{P. aeruginosa}. Therefore, a strategy to withstand the pathogen could be by impairing the zinc uptake in bacteria. Another method is to target the metallophores which are responsible for zinc uptake. By bio-engineering them anti-microbials could be delivered straight into the cell. 

Other alternative treatment options rely on light to damage the biofilm - Photothermal and Photodynamic Therapy. Here light can either be used to create localized heat spots which can cause damage to the biofilm, or create ROS which can cause oxidative stress. These methods could also be supplemented with phage therapy. All of these treatment options are still in the development phase and not ready for commercial use. This is also due to the complexity when using them on humans as one could harm unintended cells.  
