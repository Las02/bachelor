Kommentare
- Måske lidt mere background om emph{P. aeruginosa}
- Vi skal nok bedre forklare i purpose hvorfor vi kigger på biofilm og efflux, det skal i hvert fald beskrives et sted
- Flere billeder?
- Vi skal finde ud af hvordan layoutet skal være



{Pseudomonas aeruginsa}'s virulence factors; the efflux pumps and biofilm formation, and zinc homeostatis.



" A description of why this theme is microbiologically and / or socially
interesting ending with a specific section on the purpose of your report (typically one page in
total). You must define a purpose within the subject yourself.  "


\section{old Background and purpose}
\textit{P. aeruginosa} is an gram-negative opportunistic pathogen, which is commonly found found in a variety of different places such as; soil, water, vegetation, as part of the skin-microbiota.
It's large genome makes it able to host a wide range of genes for the adaptation to different envierments. It can be found both in its planktonic state and in it's biofilm lifestile where can sit on both biotic surfaces and abiotic surfaces. Furtheremore \textit{P. aeruginosa} has a large set of genes for stress response. I can for example survive on very dry places  and sometimes even survive alcohol disinfections.
Especially on medical devices \textit{P. aeruginosa} contamination can pose serious problems as it can cause nosocomial infections. 
\textit{P. aeruginosa} is not of risk to most people, but patients who are immuno-compromised are of risk to be infected. It was estimated that in 2017 32.600 people in the US where infected with \textit{P. aeruginosa}, and around 2.700 died from the infection \cite{MULTIDRUG-RESISTANTAERUGINOSA}.
Infection with \textit{P. aeruginosa} are often extremely hard to treat, as the bacteria is resistant to available treatments options such as antibiotics. 
The two major factors to \textit{P. aeruginosa's} antibiotic resistance, are biofilm and various Multi Drug Resistence efflux pumps. The efflux system, which is a series of membrane bound proteins, can export molecules from inside the cell. This can enable the cell to pump out any anti-microbial compounds from the cell, thereby increasing resistance. 

Biofilm on the other hand also promotes antibiotic resistance by blocking antibiotic access to the cells. Furthermore the biofilm can also support communities of bacteria, which in turn could harbor other pathogenic bacteria. These communities could also act as hot spots for gene transfer, thereby propelling the antibiotic genes to other species, further complicating the challenges antibiotic resistance pose.

Based on these factors \textit{Pseudomonas aeruginosa} has been listed by the World Health Organization (WHO) as one of the most life-threatening pathogenic bacteria \cite{Pachori2019EmergenceReview}. It is therefore highly relevant to find possible non-antibiotic treatments for \textit{P. aeruginosa} infections.\\ 
 The purpose of this paper is to describe novel ways of treating \textit{P. aeruginosa} infections without the use of antibiotics. To do this we will take hold in \textit{P. aeruginosa's} systems for maintaining zinc homeostatis.  as many of \textit{P. aeruginosa's} virulence factors are zinc dependent.  By inhibiting some of \textit{P. aeruginosa's} virulence factors. Thereby decreasing \textit{P. aeruginosa's} pathogenicity while not putting as big a selective pressure for resistance.  Furthermore we will focus on \textit{P. aeruginosa's} biofilm formation, as it is one of the key virulence factors in \textit{P. aeruginosa} infections.







\section{gamle purpose}
\textit{P. aeruginosa} is an gram-negative opportunistic pathogen, which under specific conditions can cause infections can 


In 2017 the gram-negative bacteria \textit{Pseudomonas aeruginosa} was listed as one of the most life-threatening pathogenic bacteria for the Research and Development of new antibiotics by the World Health Organization (WHO). It is therefore of high priority for the WHO to find possible treatments against this bacterium \cite{Pachori2019EmergenceReview}. \textit{P. aeruginosa} is versatile and under specific conditions an opportunistic pathogen. The versatility comes from  the different efflux and influx systems which enables it to survive under different conditions - in turn making it more difficult to treat \cite{Mastropasqua2018EfficientLung} \cite{Lyczak2000EstablishmentOpportunist}. Further it can utilize different energy sources, which makes it possible for the bacteria to colonize many different surfaces. \textit{P. aeruginosa} is therefore found in a variety of places such as; soil, water, vegetation, and as part of the skin-microbiota. 

The bacteria's success in colonizing surfaces is also dependent on biofilm. Biofilm also has the ability of greatly decreasing the permability of antibiotics, making infections with \textit{P. aeruginosa} difficult to treat.



Another important virulence factor is the ability to create biofilms \cite{Al-Bakri2019Photothermal-inducedBiofilm}. In this report the definition of a virulence factor is a method for measuring the pathogenicity of a bacteria and their relative ability to cause disease \cite{Madigan2022BrockMicroorganisms}. Almost every infection with \textit{P. aeruginosa} is due to a reduced immune system function, such as patients with cancer, post-surgery, severe burns, infected with HIV and Cystic Fibrosis (CF) \cite{Thi2020PseudomonasBiofilms}. The reduced immune system function makes it possible for \textit{P. aeruginosa}'s proteases to compromise physical barriers in the host as well as manipulate the host pathogen interactions \cite{Lyczak2000EstablishmentOpportunist} \cite{Pachori2019EmergenceReview}. All of the mentioned virulence factors combined with a low permeability of the outer membrane makes \textit{P. aeruginosa} more antibiotic resistant - THOUGH THE PROBABLY BIGGEST CONTRIBUTOR IS THE SELECTION PRESSURE MADE BY THE OVERUSE OF ANTIBIOTICS. 
The combination of efflux and influx systems with the biofilm creation makes it possible for \textit{P. aeruginosa} to infect Cystic Fibrosis patients chronically. Here the influx system is especially important as it makes \textit{P. aeruginosa} able to obtain the necessary trace metals such as zinc \cite{Mastropasqua2018EfficientLung}.

Cystic fibrosis is a genetic illness, causing long-term infections in the lungs, disabling the lung function over time. It has been estimated by the Cystic Fibrosis Foundation, that approximately 70.000 people worldwide suffer from this disease \cite{CysticFibrosisFoundationBasicsProtein} \cite{cysticfibrosisnewstoday.com2016CysticStatistics}. CF is a genetically determined disease, caused by the gene CFTR, cystic fibrosis transmembrane conductance regulator. This is a regulatory protein accountable for the sodium and chloride transport in and out of the cells, inside or surrounding the lungs as well as other organs \cite{Booth2019CysticFibrosis}. Patients with CF, have a mutation in this accountable gene, which reduces the expression of the CFTR protein. This brings an imbalance to the transmembrane system, which leads to accumulation of sodium, chloride, other ions and water in the cells \cite{CysticFibrosisFoundationBasicsProtein}. As a side effect, mucus is produced and accumulates in the lungs, aside from this the lungs are prone for other types of lung infections, i.e. a \textit{Pseudomonas aeruginosa} infection. 
Till this day the treatment against \textit{P. aeruginosa} infections has been with ineffective antibiotics, which has resulted in even more resistance and is therefore more difficult to treat. 

This project, the focus will be the virulence factors related to \textit{Pseudomonas aeruginosa} infection. Furthermore, alternative treatment options of how to treat \textit{P. aeruginosa} will be discussed. 





\textit{P. aeruginosa} is an gram-negative opportunistic pathogen, which is commonly found found in a variety of different places such as; soil, water, vegetation, as part of the skin-microbiota.

and most notably the pathogen can also be found on medical equipment in hospitals. The di


It's large genome makes it able to host a wide range of genes for the adaptation to different envierments.  It can be found both in its planktonic state and in it's biofilm lifestile where can sit on both biotic surfaces and abiotic surfaces. 


In 2017 the gram-negative bacteria \textit{Pseudomonas aeruginosa} was listed as one of the most life-threatening pathogenic bacteria for the Research and Development of new antibiotics by the World Health Organization (WHO). It is therefore of high priority for the WHO to find possible treatments against this bacterium \cite{Pachori2019EmergenceReview}. \textit{P. aeruginosa} is versatile and under specific conditions an opportunistic pathogen. The versatility comes from  the different efflux and influx systems which enables it to survive under different conditions - in turn making it more difficult to treat \cite{Mastropasqua2018EfficientLung} \cite{Lyczak2000EstablishmentOpportunist}. Further it can utilize different energy sources, which makes it possible for the bacteria to colonize many different surfaces. \textit{P. aeruginosa} is therefore found in a variety of places such as; soil, water, vegetation, and as part of the skin-microbiota. 


\textit{P. aeruginosa} is an gram-negative opportunistic pathogen. \textit{Pseudomonas aeruginosa} predominately infects immunocompromised patients, such as patients with HIV, severe burns or Cystic Fibrosis (CF)\cite{Thi2020PseudomonasBiofilms}. Infection with PA are often extremely hard to treat as the bacteria is extremely resistant to available treatments such as antibiotics. Therefore \textit{Pseudomonas aeruginosa} has been listed by the World Health Organization (WHO) as one of the most life-threatening pathogenic bacteria \cite{Pachori2019EmergenceReview}. It is therefore highly relevant to find possible treatments for PA infections.\\

One of the most important virulence factor of \textit{Pseudomonas aeruginosa} is it ability to create biofilms \cite{Al-Bakri2019Photothermal-inducedBiofilm}.



Many of PA species have these efflux pumps encoded in their core genome, where are evolved to have a function as a stress response to maintain homeostatis. Furtheremore some 